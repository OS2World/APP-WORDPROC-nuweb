\documentstyle{article}

\setlength{\oddsidemargin}{0in}
\setlength{\evensidemargin}{0in}
\setlength{\topmargin}{0in}
\addtolength{\topmargin}{-\headheight}
\addtolength{\topmargin}{-\headsep}
\setlength{\textheight}{8.9in}
\setlength{\textwidth}{6.5in}
\setlength{\marginparwidth}{0.5in}

\title{Nuweb \\ A Simple Literate Programming Tool}
\date{Preston Briggs \\ \sl preston@cs.rice.edu}
\author{(Version 0.87)}

\begin{document}
\maketitle

\section{Introduction}

In 1984, Knuth introduced the idea of {\em literate programming\/} and
described a pair of tools to support the practise~\cite{knuth:84}.
His approach was to combine Pascal code with \TeX\ documentation to
produce a new language, \verb|WEB|, that offered programmers a superior
approach to programming. He wrote several programs in \verb|WEB|,
including \verb|weave| and \verb|tangle|, the programs used to support
literate programming.
The idea was that a programmer wrote one document, the web file, that
combined documentation (written in \TeX~\cite{texbook}) with code
(written in Pascal).

Running \verb|tangle| on the web file would produce a complete
Pascal program, ready for compilation by an ordinary Pascal compiler.
The primary function of \verb|tangle| is to allow the programmer to
present elements of the program in any desired order, regardless of
the restrictions imposed by the programming language. Thus, the
programmer is free to present his program in a top-down fashion,
bottom-up fashion, or whatever seems best in terms of promoting
understanding and maintenance.

Running \verb|weave| on the web file would produce a \TeX\ file, ready
to be processed by \TeX\@. The resulting document included a variety of
automatically generated indices and cross-references that made it much
easier to navigate the code. Additionally, all of the code sections
were automatically pretty printed, resulting in a quite impressive
document. 

Knuth also wrote the programs for \TeX\ and {\small\sf METAFONT}
entirely in \verb|WEB|, eventually publishing them in book
form~\cite{tex:program,metafont:program}. These are probably the
largest programs ever published in a readable form.

Inspired by Knuth's example, many people have experimented with
\verb|WEB|\@. Some people have even built web-like tools for their
own favorite combinations of programming language and typesetting
language. For example, \verb|CWEB|, Knuth's current system of choice,
works with a combination of C (or C++) and \TeX~\cite{levy:90}.
Another system, FunnelWeb, is independent of any programming language
and only mildly dependent on \TeX~\cite{funnelweb}. Inspired by the
versatility of FunnelWeb and by the daunting size of its
documentation, I decided to write my own, very simple, tool for
literate programming.%
\footnote{There is another system similar to
mine, written by Norman Ramsey, called {\em noweb}~\cite{noweb}. It
perhaps suffers from being overly Unix-dependent and requiring several
programs to use. On the other hand, its command syntax is very nice.
In any case, nuweb certainly owes its name and a number of features to
his inspiration.}


\section{Nuweb}

Nuweb works with any programming language and \LaTeX~\cite{latex}. I
wanted to use \LaTeX\ because it supports a multi-level sectioning
scheme and has facilities for drawing figures. I wanted to be able to
work with arbitrary programming languages because my friends and I
write programs in many languages (and sometimes combinations of
several languages), {\em e.g.,} C, Fortran, C++, yacc, lex, Scheme,
assembly, Postscript, and so forth. The need to support arbitrary
programming languages has many consequences:
\begin{description}
\item[No pretty printing] Both \verb|WEB| and \verb|CWEB| are able to
  pretty print the code sections of their documents because they
  understand the language well enough to parse it. Since we want to use
  {\em any\/} language, we've got to abandon this feature.
\item[No index of identifiers] Because \verb|WEB| knows about Pascal,
  it is able to construct an index of all the identifiers occurring in
  the code sections (filtering out keywords and the standard type
  identifiers). Unfortunately, this isn't as easy in our case. We don't
  know what an identifiers looks like in each language and we certainly
  don't know all the keywords. (On the other hand, see the end of
  Section~1.3)
\end{description}
Of course, we've got to have some compensation for our losses or the
whole idea would be a waste. Here are the advantages I can see:
\begin{description}
\item[Simplicity] The majority of the commands in \verb|WEB| are
  concerned with control of the automatic pretty printing. Since we
  don't pretty print, many commands are eliminated. A further set of
  commands is subsumed by \LaTeX\  and may also be eliminated. As a
  result, our set of commands is reduced to only four members (explained
  in the next section). This simplicity is also reflected in
  the size of this tool, which is quite a bit smaller than the tools
  used with other approaches.
\item[No pretty printing] Everyone disagrees about how their code
  should look, so automatic formatting annoys many people. One approach
  is to provide ways to control the formatting. Our approach is simpler
  -- we perform no automatic formatting and therefore allow the
  programmer complete control of code layout.
\item[Control] We also offer the programmer complete control of the
  layout of his output files (the files generated during tangling). Of
  course, this is essential for languages that are sensitive to layout;
  but it is also important in many practical situations, {\em e.g.,}
  debugging.
\item[Speed] Since nuweb doesn't do to much, the nuweb tool runs
  quickly. I combine the functions of \verb|tangle| and \verb|weave| into
  a single program that performs both functions at once.
\item[Page numbers] Inspired by the example of noweb, nuweb refers to
  all scraps by page number to simplify navigation. If there are
  multiple scraps on a page (say page~17), they are distinguished by
  lower-case letters ({\em e.g.,} 17a, 17b, and so forth).
\item[Multiple file output] The programmer may specify more than one
  output file in a single nuweb file. This is required when constructing
  programs in a combination of languages (say, Fortran and C)\@. It's also
  an advantage when constructing very large programs that would require
  a lot of compile time.
\end{description}
This last point is very important. By allowing the creation of
multiple output files, we avoid the need for monolithic programs.
Thus we support the creation of very large programs by groups of
people. 

A further reduction in compilation time is achieved by first
writing each output file to a temporary location, then comparing the
temporary file with the old version of the file. If there is no
difference, the temporary file can be deleted. If the files differ,
the old version is deleted and the temporary file renamed. This
approach works well in combination with \verb|make| (or similar tools),
since \verb|make| will avoid recompiling untouched output files.


\section{Writing Nuweb}

The bulk of a nuweb file will be ordinary \LaTeX\@. In fact, any \LaTeX\
file can serve as input to nuweb and will be simply copied through
unchanged to the \verb|.tex| file -- unless a nuweb command is
discovered. All nuweb commands begin with an ``at-sign'' (\verb|@|).
Therefore, a file without at-signs will be copied unchanged.
Nuweb commands are used to specify {\em output files,} define 
{\em macros,} and delimit {\em scraps}. These are the basic features
of interest to the nuweb tool -- all else is simply text to be copied
to the \verb|.tex| file.

\subsection{The Major Commands}

Files and macros are defined with the following commands:
\begin{description}
\item[{\tt @o} {\em file-name flags scrap\/}] Output a file. The file name is
  terminated by whitespace.
\item[{\tt @d} {\em macro-name scrap\/}] Define a macro. The macro name
  is terminated by a return or the beginning of a scrap.
\end{description}
A specific file may be specified several times, with each definition
being written out, one after the other, in the order they appear.
The definitions of macros may be similarly divided.

\subsubsection{Scraps}

Scraps have specific begin markers and end markers to allow precise
control over the contents and layout. Note that any amount of
whitespace (including carriage returns) may appear between a name and
the beginning of a scrap.
\begin{description}
\item[\tt @\{{\em anything\/}@\}] where the scrap body includes every
  character in {\em anything\/} -- all the blanks, all the tabs, all the
  carriage returns.
\end{description}
Inside a scrap, we may invoke a macro.
\begin{description}
\item[\tt @<{\em macro-name\/}@>] Causes the macro 
  {\em macro-name\/} to be expanded inline as the code is written out
  to a file. It is an error to specify recursive macro invocations.
\end{description}
Note that macro names may be abbreviated, either during invocation or
definition. For example, it would be very tedious to have to
repeatedly type the macro name
\begin{quote}
\verb|@d Check for terminating at-sequence and return name if found|
\end{quote}
Therefore, we provide a mechanism (stolen from Knuth) of indicating
abbreviated names.
\begin{quote}
\verb|@d Check for terminating...|
\end{quote}
Basically, the programmer need only type enough characters to uniquely
identify the macro name, followed by three periods. An abbreviation
may even occur before the full version; nuweb simply preserves the
longest version of a macro name. Note also that blanks and tabs are
insignificant in a macro name; any string of them are replaced by a
single blank.

When scraps are written to an output or \verb|.tex| file, tabs are
expanded into spaces by default. Currently, I assume tab stops are set
every eight characters. Furthermore, when a macro is expanded in a scrap,
the body of the macro is indented to match the indentation of the
macro invocation. Therefore, care must be taken with languages 
({\em e.g.,} Fortran) that are sensitive to indentation.
These default behaviors may be changed for each output file (see
below).

\subsubsection{Flags}

When defining an output file, the programmer has the option of using
flags to control output of a particular file. The flags are intended
to make life a little easier for programmers using certain languages.
They introduce little language dependences; however, they do so only
for a particular file. Thus it is still easy to mix languages within a
single document. There are three ``per-file'' flags:
\begin{description}
\item[\tt -d] Forces the creation of \verb|#line| directives in the
  output file. These are useful with C (and sometimes C++ and Fortran) on
  many Unix systems since they cause the compiler's error messages to
  refer to the web file rather than the output file. Similarly, they
  allow source debugging in terms of the web file.
\item[\tt -i] Suppresses the indentation of macros. That is, when a
  macro is expanded in a scrap, it will {\em not\/} be indented to
  match the indentation of the macro invocation. This flag would seem
  most useful for Fortran programmers.
\item[\tt -t] Suppresses expansion of tabs in the output file. This
  feature seems important when generating \verb|make| files.
\end{description}


\subsection{The Minor Commands}

We have two very low-level utility commands that may appear anywhere
in the web file.
\begin{description}
\item[\tt @@] Causes a single ``at sign'' to be copied into the output.
\item[\tt @i {\em file-name\/}] Includes a file. Includes may be
  nested, though there is currently a limit of 10~levels. The file name
  should be complete (no extension will be appended) and should be
  terminated by a carriage return.
\end{description}
Finally, there are three commands used to create indices to the macro
names, file definitions, and user-specified identifiers.
\begin{description}
\item[\tt @f] Create an index of file names.
\item[\tt @m] Create an index of macro name.
\item[\tt @u] Create an index of user-specified identifiers.
\end{description}
I usually put these in their own section
in the \LaTeX\ document.

Identifiers must be explicitely specified for inclusion in the
\verb|@u| index. By convention, each identifier is marked at the
point of its definition; all references to each identifier (inside
scraps) will be discovered automatically. To ``mark'' an identifier
for inclusion in the index, we must mention it at the end of a scrap.
For example,
\begin{quote}
\begin{verbatim}
@d a scrap @{
Let's pretend we're declaring the variables FOO and BAR
inside this scrap.
@| FOO BAR @}
\end{verbatim}
\end{quote}
I've used alphabetic identifiers in this example, but any string of
characters (not including whitespace or \verb|@| characters) will do.
Therefore, it's possible to add index entries for things like
\verb|<<=| if desired. An identifier may be declared in more than one
scrap.

In the generated index, each identifier appears with a list of all the
scraps using and defining it, where the defining scraps are
distinguished by underlining. Note that the identifier doesn't
actually have to appear in the defining scrap; it just has to be in
the list of definitions at the end of a scrap.


\section{Running Nuweb}

Nuweb is invoked using the following command:
\begin{quote}
{\tt nuweb} {\em flags file-name}\ldots
\end{quote}
One or more files may be processed at a time. If a file name has no
extension, \verb|.w| will be appended. While a file name may specify a
file in another directory, the resulting \verb|.tex| file will always
be created in the current directory. For example,
\begin{quote}
{\tt nuweb /foo/bar/quux}
\end{quote}
will take as input the file \verb|/foo/bar/quux.w| and will create the
file \verb|quux.tex| in the current directory.

By default, nuweb performs both tangling and weaving at the same time.
Normally, this is not a bottleneck in the compilation process;
however, it's possible to achieve slightly faster throughput by
avoiding one or another of the default functions using command-line
flags. There are currently three possible flags:
\begin{description}
\item[\tt -t] Suppress generation of the \verb|.tex| file.
\item[\tt -o] Suppress generation of the output files.
\item[\tt -c] Avoid testing output files for change before updating them.
\end{description}
Thus, the command
\begin{quote}
\verb|nuweb -to /foo/bar/quux|
\end{quote}
would simply scan the input and produce no output at all.

There are two additional command-line flags:
\begin{description}
\item[\tt -v] For ``verbose,'' causes nuweb to write information about
  its progress to \verb|stderr|.
\item[\tt -n] Forces scraps to be numbered sequentially from~1
  (instead of using page numbers). This form is perhaps more desirable
  for small webs.
\end{description}


\section{Restrictions}

Because nuweb is intended to be a simple tool, I've established a few
restrictions. Over time, some of these may be eliminated; others seem
fundamental.
\begin{itemize}
\item The handling of errors is not completely ideal. In some cases, I
  simply warn of a problem and continue; in other cases I halt
  immediately. This behavior should be regularized.
\item I warn about references to macros that haven't been defined, but
  don't halt. This seems most convenient for development, but may change
  in the future.
\item File names and index entries should not contain any \verb|@|
  signs.
\item Macro names may be (almost) any well-formed \TeX\ string.
  It makes sense to change fonts or use math mode; however, care should
  be taken to ensure matching braces, brackets, and dollar signs.
\item Anything is allowed in the body of a scrap; however, very
  long scraps (horizontally or vertically) may not typeset well.
\item Temporary files (created for comparison to the eventual
  output files) are placed in the current directory. Since they may be
  renamed to an output file name, all the output files should be on the
  same file system as the current directory.
\item Because page numbers cannot be determined until the document has
  been typeset, we have to rerun nuweb after \LaTeX\ to obtain a clean
  version of the document (very similar to the way we sometimes have
  to rerun \LaTeX\ to obtain an up-to-date table of contents after
  significant edits).  Nuweb will warn (in most cases) when this needs
  to be done; in the remaining cases, \LaTeX\ will warn that labels
  may have changed.
\end{itemize}
Very long scraps may be allowed to break across a page if declared
with \verb|@O| or \verb|@D| (instead of \verb|@o| and \verb|@d|).
This doesn't work very well as a default, since far too many short
scraps will be broken across pages; however, as a user-controlled
option, it seems very useful.

\section{Acknowledgements}

Several people have contributed their times, ideas, and debugging
skills. In particular, I'd like to acknowledge the contributions of
Osman Buyukisik, Manuel Carriba, Adrian Clarke, Tim Harvey, Michael
Lewis, Norman Ramsey, Walter Ravenek, Rob Shillingsburg, Kayvan
Sylvan, Dominique de~Waleffe, and Scott Warren.  Of course, most of
these people would never have heard or nuweb (or many other tools)
without the efforts of George Greenwade.


\bibliographystyle{plain}
\bibliography{literate}

\end{document}
